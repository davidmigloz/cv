%!TEX TS-program = xelatex
\documentclass[]{friggeri-cv}
\usepackage{afterpage}
\usepackage{hyperref}
\usepackage{color}
\usepackage{xcolor}
\usepackage{smartdiagram}
\usepackage{fontspec}
% if you want to add fontawesome package
% you need to compile the tex file with LuaLaTeX
% References:
%   http://texdoc.net/texmf-dist/doc/latex/fontawesome/fontawesome.pdf
%   https://www.ctan.org/tex-archive/fonts/fontawesome?lang=en
\usepackage{fontawesome}
\usepackage{metalogo}
\usepackage{dtklogos}
\usepackage{enumitem}
\usepackage[utf8]{inputenc}
\usepackage{tikz}
\usetikzlibrary{mindmap,shadows}
\hypersetup{
    pdftitle={},
    pdfauthor={},
    pdfsubject={},
    pdfkeywords={},
    colorlinks=false,           % no lik border color
    allbordercolors=white       % white border color for all
}
\smartdiagramset{
    bubble center node font = \footnotesize,
    bubble node font = \footnotesize,
    % specifies the minimum size of the bubble center node
    bubble center node size = 0.5cm,
    %  specifies the minimum size of the bubbles
    bubble node size = 0.5cm,
    % specifies which is the distance among the bubble center node and the other bubbles
    distance center/other bubbles = 0.3cm,
    % sets the distance from the text to the border of the bubble center node
    distance text center bubble = 0.5cm,
    % set center bubble color
    bubble center node color = pblue,
    % define the list of colors usable in the diagram
    set color list = {lightgray, materialcyan, orange, green, materialorange, materialteal, materialamber, materialindigo, materialgreen, materiallime},
    % sets the opacity at which the bubbles are shown
    bubble fill opacity = 0.6,
    % sets the opacity at which the bubble text is shown
    bubble text opacity = 0.5,
}

\addbibresource{bibliography.bib}
\RequirePackage{xcolor}
\definecolor{pblue}{HTML}{0395DE}

\begin{document}
\header{David}{Miguel Lozano}
      
% Fake text to add separator      
\fcolorbox{white}{gray}{\parbox{\dimexpr\textwidth-2\fboxsep-2\fboxrule}{%
.....
}}

% In the aside, each new line forces a line break
\begin{aside}
  \includegraphics[scale=0.18]{img/profile.png}
  \section{Contact}
    +34 628 16 33 57
    \href{mailto:me@davidmiguel.com}{me@davidmiguel.com}
    ~  
  \section{Date of birth}
    08/11/1994
    ~
  \section{Address}
    De la Reijstraat 26
    1091 PB Amsterdam
    (Netherlands).
    ~    
  \section{Web}
    \faGlobe\ \href{http://davidmiguel.com}{davidmiguel.com}
    \faLinkedin\ \href{https://www.linkedin.com/in/davidmigloz}{/davidmigloz}
    \faFacebook\ \ \href{https://www.facebook.com/DavidMigLoz}{/davidmigloz}
    \faTwitter\ \href{https://twitter.com/DavidMigLoz}{/davidmigloz}
    \faGithub\ \href{https://github.com/davidmigloz/}{/davidmigloz}
    \faStackOverflow\ \href{http://stackoverflow.com/users/6305235/david-miguel}{/david-miguel}
    ~
  \section{Languages}
    - Spanish (native)
    - English (advanced)
    ~       
  \section{Programming\\[0.5cm]}
    \includegraphics[scale=0.9]{img/cloud.png}
    ~    
\end{aside}
~
\\ [0.8cm]
\section{Experience}
\begin{entrylist}
    \entry
    {09/13 - act.}
    {Information systems vocal}
    {\href{http://abi2burgos.es/}{Computer Engineers Association of Burgos}}
    {Responsible for developing and maintaining the institution's website, managing social profiles and organizing events (OpenWeek, Google HashCode, Information Technologies Week, etc.).\\}
    \entry
    {01/11 - act.}
    {Freelance web designer and developer}
    {\href{http://davidmiguel.com/}{davidmiguel.com}}
    {Front-end and back-end development in different projects. Using technologies like HTML5, CSS3, JavaScript, PHP, MySQL, frameworks (jQuery, Bootstrap, etc.) and content management systems (Wordpress, Joomla y Drupal).
    }
    \entry
    {08/16 - 09/16}
    {Beekeeper}
    {Julián Muñoz beekeeping}
    {Honey collection and extraction, apiaries vaccination and hives transportation in a professional beekeeping company.\\}
\end{entrylist}

\section{Projects}
\begin{entrylist}
  \entry
    {09/16 - act.}
    {GoBees - Smart apiary management}
    {\href{http://gobees.io/}{gobees.io}}
    {Platform for the management and monitoring of apiaries. At present, it is available for Android devices. Its main feature is the ability to monitor the flight activity of a honey bee colony using the built-in camera of a smartphone. It was developed as my bachelor's dissertation, obtaining a grade of 10 / 10.}
\end{entrylist}

\section{Education}
\begin{entrylist}
  \entry
    {09/12 - 02/17}
    {Degree in Computer Engineering}
    {\href{http://wwww.ubu.es/}{University of Burgos}}
    {Average mark : 8,8 / 10 (best student’s record).\\}
  \entry
    {09/15 - 06/16}
    {Degree in Computer Science - Erasmus+}
    {\href{https://www.pw.edu.pl/}{Warsaw University of Technology}}
    {Average mark: 4,6 / 5.\\}
  \entry
    {09/09 - 06/15}
    {English}
    {\href{http://eoiburgos.centros.educa.jcyl.es/}{Burgos Official Language School}}
    {Level: advanced.\\}
\end{entrylist}

\section{Certifications}
\begin{entrylist}
  \entry
    {02/2016}
    {ISTQB Foundation Level}
    {\href{http://sjsi.org/}{Stowarzyszenie Jakosci Systemow Informatycznych}}
    {\emph{International Software Testing Qualifications Board. Licencia 5325/2016.}}
\end{entrylist}

\newpage

\begin{aside}
~
~
~
  \section{\vspace{0.49cm}Expertise}
    \textbf{- Android}
    \textbf{- Arduino}
    \textbf{- C}
    \textbf{- CI}    
    \textbf{- Clojure}
    \textbf{- CSS3}
    \textbf{- CUDA}
    \textbf{- Photography}
    \textbf{- Git}
    \textbf{- GitFlow}    
    \textbf{- Gradle}
    \textbf{- HTML5}
    \textbf{- Jade}
    \textbf{- Java}
    \textbf{- JavaFX}
    \textbf{- JavaScript}
    \textbf{- jQuery}
    \textbf{- \LaTeX}
    \textbf{- Lightroom}
    \textbf{- Linux}
    \textbf{- Markdown}
    \textbf{- Matlab}
    \textbf{- Maven}
    \textbf{- MPI}
    \textbf{- MySQL}
    \textbf{- Office}
    \textbf{- OpenCV}
    \textbf{- ORM}
    \textbf{- Photoshop}
    \textbf{- PHP}
    \textbf{- Python}
    \textbf{- Realm}
    \textbf{- SQL}
    \textbf{- SQLite}    
    \textbf{- Testing}
    \textbf{- Windows}
    \textbf{- WordPress}
    \textbf{- XML}
    ~
  \section{Interests}
    \textbf{- Traveling}
    \textbf{- Photography}
    \textbf{- Tech events}    
    \textbf{- Swim}
    \textbf{- Mountaineering}
    \textbf{- Mountain bike}
    \textbf{- Guitar}    
    \textbf{- Beekeeping}
    ~
\end{aside}

\section{Courses}
\begin{entrylist}
  \entry
    {2016}
    {Android Development for Beginners}
    {\href{https://www.udacity.com/course/android-development-for-beginners--ud837}{Google}}
    {Introductory course on Android application development.}
  \entry
    {2016}
    {Psychological First Aid}
    {\href{https://www.coursera.org/learn/pap}{Autonomous University of Barcelona}}
    {Application of psychological first aid (PFA) to people affected by highly stressful situations.}
  \entry
    {2016}
    {Learning How to Learn: Powerful mental tools to help you master tough subjects}
    {\href{https://coursera.org/learn/learning-how-to-learn/}{University of California San Diego}}
    {Learning techniques, neuroscience and procrastination.}
  \entry
    {2015}
    {Wedding photography}
    {\href{http://www.carlosvaquero.es/}{Carlos Vaquero}}
    {Introductory course to wedding photography.}
  \entry
    {2015}
    {Development of mobile applications with Android}
    {\href{http://agilecyl.org/}{agilecyl}}
    {Introductory course on Android application development.}
  \entry
    {2014}
    {First Aid}
    {\href{http://www.cruzroja.es/}{Red Cross}}
    {Basic first aid techniques.}
  \entry
    {2014}
    {Digital marketing}
    {\href{https://www.google.es/landing/activate/}{Google Actívate}} 
    {Fundamentals of technology, SEO, SEM, web analytics, e-commerce, RRSS, mobile devices, entrepreneurship, etc.}
  \entry
    {2014}
    {Agile web development with Python and Django}
    {\href{http://www.ceeiburgos.es/}{CEEI Burgos}} 
    {Introductory course to agile web development with Python and Django.}    
  \entry
    {2013}
    {Design, layout and web developing}
    {\href{http://abi2burgos.es/}{Computer Engineers Association of Burgos}}   
    {How to develop a web page from scratch.}     
  \entry
    {2012}
    {Introduction to web development}
    {\href{https://www.ua.es/}{University of Alicante}}  
    {Introductory course to web development: HTML, CSS and JavaScript.}     
  \entry
    {2011}
    {A year of photography}
    {\href{http://josebruiz.com/}{José Benito Ruiz}}    
    {One-year advanced digital photography course.}      
\end{entrylist}

\section{Awards}
\begin{entrylist}
  \entry
    {01/2017}
    {YUZZ "Young people with ideas"}
    {\href{http://yuzz.org.es/}{Santander Universidades}}
    {Selected for the Yuzz program with GoBees project. Yuzz provides advice, spaces for coworking and training to young entrepreneurs who want to develop an innovative idea.}
  \entry
    {12/2016}
    {Market-Oriented Prototypes Award}
    {\href{http://wwww.ubu.es/te-interesa/convocatoria-prototipos-orientados-al-mercado-curso-2016-2017}{University of Burgos - OTRI}}
    {GoBees won one of the three awards for developing prototypes with possibilities of being commercialized.}
  \entry
    {12/2016}
    {Student collaboration scholarship with university departments\\}
    {\href{http://www.mecd.gob.es/}{Ministry of Education, Culture and Sports}}
    {Winner of one of the fifteen collaboration scholarship assigned to the University of Burgos for students to get experience of research tasks. Department: Civil Engineering.}  
\end{entrylist}

\section{Volunteering}
\begin{entrylist}
  \entry
    {09/2014}
    {Workcamp Mikulov (Czech Republic)}
    {\href{https://www.inexsda.cz/}{INEX-SDA}}
    {Several environmental tasks were carried out.}
  \entry
    {09/2013}
    {Workcamp Clausthal-Zellerfeld (Germany)}
    {\href{http://www.ijgd.de/}{IJGD}}
    {Several forestry tasks were carried out.}
\end{entrylist}
\end{document}
