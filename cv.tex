%!TEX TS-program = xelatex
\documentclass[]{friggeri-cv}
\usepackage[spanish]{babel}
\usepackage{afterpage}
\usepackage{hyperref}
\usepackage{color}
\usepackage{xcolor}
\usepackage{smartdiagram}
\usepackage{fontspec}
% if you want to add fontawesome package
% you need to compile the tex file with LuaLaTeX
% References:
%   http://texdoc.net/texmf-dist/doc/latex/fontawesome/fontawesome.pdf
%   https://www.ctan.org/tex-archive/fonts/fontawesome?lang=en
\usepackage{fontawesome}
\usepackage{metalogo}
\usepackage{dtklogos}
\usepackage{enumitem}
\usepackage[utf8]{inputenc}
\usepackage{tikz}
\usetikzlibrary{mindmap,shadows}
\hypersetup{
    pdftitle={},
    pdfauthor={},
    pdfsubject={},
    pdfkeywords={},
    colorlinks=false,           % no lik border color
    allbordercolors=white       % white border color for all
}
\smartdiagramset{
    bubble center node font = \footnotesize,
    bubble node font = \footnotesize,
    % specifies the minimum size of the bubble center node
    bubble center node size = 0.5cm,
    %  specifies the minimum size of the bubbles
    bubble node size = 0.5cm,
    % specifies which is the distance among the bubble center node and the other bubbles
    distance center/other bubbles = 0.3cm,
    % sets the distance from the text to the border of the bubble center node
    distance text center bubble = 0.5cm,
    % set center bubble color
    bubble center node color = pblue,
    % define the list of colors usable in the diagram
    set color list = {lightgray, materialcyan, orange, green, materialorange, materialteal, materialamber, materialindigo, materialgreen, materiallime},
    % sets the opacity at which the bubbles are shown
    bubble fill opacity = 0.6,
    % sets the opacity at which the bubble text is shown
    bubble text opacity = 0.5,
}

\addbibresource{bibliography.bib}
\RequirePackage{xcolor}
\definecolor{pblue}{HTML}{0395DE}

\begin{document}
\header{David}{Miguel Lozano}
      
% Fake text to add separator      
\fcolorbox{white}{gray}{\parbox{\dimexpr\textwidth-2\fboxsep-2\fboxrule}{%
.....
}}

% In the aside, each new line forces a line break
\begin{aside}
  \includegraphics[scale=0.18]{img/profile.png}
  \section{Dirección}
    Avda. Casa la Vega 
    43 - 5ºC, 09007 
    Burgos, España.
    ~
  \section{Contacto}
    +34 628 16 33 57
    \href{mailto:me@davidmiguel.com}{me@davidmiguel.com}
    ~
  \section{Web}
    \faGlobe\ \href{http://davidmiguel.com}{davidmiguel.com}
    \faLinkedin\ \href{https://www.linkedin.com/in/davidmigloz}{/davidmigloz}
    \faFacebook\ \ \href{https://www.facebook.com/DavidMigLoz}{/davidmigloz}
    \faTwitter\ \href{https://twitter.com/DavidMigLoz}{/davidmigloz}
    \faGithub\ \href{https://github.com/davidmigloz/}{/davidmigloz}
    \faStackOverflow\ \href{http://stackoverflow.com/users/6305235/david-miguel}{/david-miguel}
    ~
  \section{Idiomas}
    \textbf{- Español (nativo)}
    \textbf{- Inglés (avanzado)}
    ~       
  \section{Habilidades}
    \textbf{- Organización}
    \textbf{- Esfuerzo}
    \textbf{- Responsabilidad}
    \textbf{- Curiosidad}
    \textbf{- Iniciativa}
    ~ 
  % use  \hspace{} or \vspace{} to change bubble size, if needed
  \section{Programación}
  \\ [0.5cm]
    \includegraphics[scale=0.9]{img/cloud.png}
    ~    
\end{aside}
~
\\ [0.8cm]
\section{Experiencia}
\begin{entrylist}
    \entry
    {09/13 - act.}
    {Vocal de Sistemas de Información}
    {\href{http://abi2burgos.es/}{Asoc. Burgalesa de Ingenieros Informáticos}}
    {Encargado del desarrollo y mantenimiento de la página web de la institución, de la gestión de los perfiles sociales y de la organización de eventos (OpenWeek, Google HashCode, etc.).\\}
    \entry
    {01/11 - act.}
    {Diseñador y desarrollador web freelance}
    {\href{http://davidmiguel.com/}{davidmiguel.com}}
    {Desarrollo \textit{front-end} y \textit{back-end} en diferentes proyectos. Mediante tecnologías como HTML5, CSS3, JavaScript, PHP, MySQL, \textit{frameworks} (jQuery, Bootstrap, etc.) y gestores de contenidos (Wordpress, Joomla y Drupal).
    }
    \entry
    {08/16 - 09/16}
    {Apicultor}
    {Apícola Julián Muñoz}
    {Durante un mes estuve realizando tareas de recolección y extracción de miel, vacunación y transporte de colmenas en esta empresa salmantina de apicultura profesional.\\}
\end{entrylist}

\section{Proyectos}
\begin{entrylist}
  \entry
    {09/16 - act.}
    {GoBees - Gestión inteligente de colmenares}
    {\href{http://gobees.io/}{gobees.io}}
    {Plataforma para la gestión y monitorización de colmenares. Actualmente está disponible para dispositivos Android. Su característica estrella es la monitorización de la actividad de vuelo de una colmena mediante la cámara de un \emph{smartphone}. Fue desarrollado como Trabajo Fin de Grado, obteniendo una calificación de 10.}
\end{entrylist}

\section{Educación}
\begin{entrylist}
  \entry
    {09/12 - 02/17}
    {Grado en Ingeniería Informática}
    {\href{http://wwww.ubu.es/}{Universidad de Burgos}}
    {Nota media: 8,8 / 10 (nota más alta de la promoción).\\}
  \entry
    {09/15 - 06/16}
    {Degree in Computer Science - Erasmus+}
    {\href{https://www.pw.edu.pl/}{Politechnika Warszawska}}
    {Nota media: 4,6 / 5.\\}
  \entry
    {09/09 - 06/15}
    {Inglés}
    {\href{http://eoiburgos.centros.educa.jcyl.es/}{Escuela Oficial de Idiomas de Burgos}}
    {Nivel: avanzado.\\}
\end{entrylist}

\section{Certificaciones}
\begin{entrylist}
  \entry
    {02/2016}
    {ISTQB Foundation Level}
    {\href{http://sjsi.org/}{Stowarzyszenie Jakosci Systemow Informatycznych}}
    {\emph{International Software Testing Qualifications Board. Licencia 5325/2016.}}
\end{entrylist}

\newpage

\begin{aside}
~
~
~
  \section{\vspace{0.49cm}Competencias}
    \textbf{- Android}
    \textbf{- Arduino}
    \textbf{- C}
    \textbf{- CI}    
    \textbf{- Clojure}
    \textbf{- CSS3}
    \textbf{- CUDA}
    \textbf{- Fotografía}
    \textbf{- Git}
    \textbf{- GitFlow}    
    \textbf{- Gradle}
    \textbf{- HTML5}
    \textbf{- Jade}
    \textbf{- Java}
    \textbf{- JavaFX}
    \textbf{- JavaScript}
    \textbf{- jQuery}
    \textbf{- \LaTeX}
    \textbf{- Lightroom}
    \textbf{- Linux}
    \textbf{- Markdown}
    \textbf{- Matlab}
    \textbf{- Maven}
    \textbf{- MPI}
    \textbf{- MySQL}
    \textbf{- Office}
    \textbf{- OpenCV}
    \textbf{- ORM}
    \textbf{- Photoshop}
    \textbf{- PHP}
    \textbf{- Python}
    \textbf{- Realm}
    \textbf{- SQL}
    \textbf{- SQLite}    
    \textbf{- Testing}
    \textbf{- Windows}
    \textbf{- WordPress}
    \textbf{- XML}
    ~
  \section{Aficiones}
    \textbf{- Viajar}
    \textbf{- Fotografía}
    \textbf{- Guitarra}
    \textbf{- Natación}
    \textbf{- Montañismo}
    \textbf{- Mountain bike}    
    \textbf{- Apicultura}
    \textbf{- Eventos tecnológicos}
    ~
\end{aside}

\section{Cursos}
\begin{entrylist}
  \entry
    {2016}
    {Android Development for Beginners}
    {\href{https://www.udacity.com/course/android-development-for-beginners--ud837}{Google}}
    {Curso introductorio al desarrollo de aplicaciones Android.}
  \entry
    {2016}
    {Primeros Auxilios Psicológicos}
    {\href{https://www.coursera.org/learn/pap}{Universitat Autònoma de Barcelona}}
    {Aplicación de primeros auxilios psicológicos (PAP) a personas afectadas por situaciones altamente estresantes.}
  \entry
    {2016}
    {Learning How to Learn: Powerful mental tools to help you master tough subjects}
    {\href{https://coursera.org/learn/learning-how-to-learn/}{Universidad de California en San Diego}}
    {Técnicas de aprendizaje, neurociencia y procrastinación.}
  \entry
    {2015}
    {Fotografía de bodas}
    {\href{http://www.carlosvaquero.es/}{Carlos Vaquero}}
    {Curso introductorio a la fotografía de bodas.}
  \entry
    {2015}
    {Desarrollo de aplicaciones móviles con Android}
    {\href{http://agilecyl.org/}{agilecyl}}
    {Curso introductorio al desarrollo de aplicaciones Android.}
  \entry
    {2014}
    {Primeros Auxilios}
    {\href{http://www.cruzroja.es/}{Cruz Roja}}
    {Técnicas básicas de Primeros Auxilios.}
  \entry
    {2014}
    {Marketing Digital}
    {\href{https://www.google.es/landing/activate/}{Google Actívate}} 
    {Fundamentos de tecnología, SEO, SEM, analítica web, comercio electrónico, RRSS, dispositivos móviles, emprendimiento, etc.}
  \entry
    {2014}
    {Desarrollo web ágil con Python y Django}
    {\href{http://www.ceeiburgos.es/}{CEEI Burgos}} 
    {Curso introductorio al desarrollo web ágil con Python y Django.}    
  \entry
    {2013}
    {Diseño, maquetación y programación web}
    {\href{http://abi2burgos.es/}{Asoc. Burgalesa de Ing. Informáticos}}   
    {Cómo desarrollar una página web desde cero.}     
  \entry
    {2012}
    {Introducción al desarrollo web}
    {\href{https://www.ua.es/}{Universidad de Alicante}}  
    {Curso introductorio al desarrollo web: HTML, CSS y JavaScript.}     
  \entry
    {2011}
    {Un año de fotografía}
    {\href{http://josebruiz.com/}{José Benito Ruiz}}    
    {Curso avanzado de fotografía digital de un año de duración.}      
\end{entrylist}

\section{Reconocimientos}
\begin{entrylist}
  \entry
    {01/2017}
    {YUZZ "Jóvenes con ideas"}
    {\href{http://yuzz.org.es/}{Santander Universidades}}
    {Seleccionado para el programa Yuzz con el proyecto GoBees. Yuzz brinda asesoramiento, espacios de coworking y formación a jóvenes emprendedores que quieran desarrollar una idea innovadora.}
  \entry
    {12/2016}
    {Premio Prototipos Orientados al Mercado}
    {\href{http://wwww.ubu.es/te-interesa/convocatoria-prototipos-orientados-al-mercado-curso-2016-2017}{Universidad de Burgos - OTRI}}
    {GoBees obtuvo uno de los tres premios para la obtención de prototipos con posibilidades de ser comercializados en el mercado.}
  \entry
    {12/2016}
    {Beca de colaboración de estudiantes con departamentos universitarios\\}
    {\href{http://www.mecd.gob.es/}{Ministerio de Educación, Cultura y Deporte}}
    {Ganador de una de las 15 becas de colaboración asignadas a la Universidad de Burgos para la toma de contacto con tareas de investigación. Departamento: Ingeniería Civil.}  
\end{entrylist}

\section{Voluntariados}
\begin{entrylist}
  \entry
    {09/2014}
    {Workcamp Mikulov (Rep. Checa)}
    {\href{https://www.inexsda.cz/}{INEX-SDA}}
    {Se realizaron diversos trabajos medioambientales.}
  \entry
    {09/2013}
    {Workcamp Clausthal-Zellerfeld (Alemania)}
    {\href{http://www.ijgd.de/}{IJGD}}
    {Se realizaron diversos trabajos forestales.}
\end{entrylist}
\end{document}
